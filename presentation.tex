%%%%%%%%%%%%%%%%%%%%%%%%%%%%%%%%%%%%%%%%%
% Beamer Presentation
% LaTeX Template
% Version 1.0 (10/11/12)
%
% This template has been downloaded from:
% http://www.LaTeXTemplates.com
%
% License:
% CC BY-NC-SA 3.0 (http://creativecommons.org/licenses/by-nc-sa/3.0/)
%
%%%%%%%%%%%%%%%%%%%%%%%%%%%%%%%%%%%%%%%%%

%----------------------------------------------------------------------------------------
%	PACKAGES AND THEMES
%----------------------------------------------------------------------------------------

\documentclass[12pt]{beamer}

\mode<presentation> {

    % The Beamer class comes with a number of default slide themes
    % which change the colors and layouts of slides. Below this is a list
    % of all the themes, uncomment each in turn to see what they look like.

    %\usetheme{default}
    %\usetheme{AnnArbor}
    %\usetheme{Antibes}
    %\usetheme{Bergen}
    %\usetheme{Berkeley}
    %\usetheme{Berlin}
    %\usetheme{Boadilla}
    %\usetheme{CambridgeUS}
    %\usetheme{Copenhagen}
    %\usetheme{Darmstadt}
    %\usetheme{Dresden}
    %\usetheme{Frankfurt}
    %\usetheme{Goettingen}
    %\usetheme{Hannover}
    %\usetheme{Ilmenau}
    %\usetheme{JuanLesPins}
    %\usetheme{Luebeck}
    \usetheme{Madrid}
    %\usetheme{Malmoe}
    %\usetheme{Marburg}
    %\usetheme{Montpellier}
    %\usetheme{PaloAlto}
    %\usetheme{Pittsburgh}
    %\usetheme{Rochester}
    %\usetheme{Singapore}
    %\usetheme{Szeged}
    %\usetheme{Warsaw}

    % As well as themes, the Beamer class has a number of color themes
    % for any slide theme. Uncomment each of these in turn to see how it
    % changes the colors of your current slide theme.

    %\usecolortheme{albatross}
    %\usecolortheme{beaver}
    %\usecolortheme{beetle}
    %\usecolortheme{crane}
    %\usecolortheme{dolphin}
    %\usecolortheme{dove}
    %\usecolortheme{fly}
    %\usecolortheme{lily}
    %\usecolortheme{orchid}
    %\usecolortheme{rose}
    %\usecolortheme{seagull}
    %\usecolortheme{seahorse}
    %\usecolortheme{whale}
    %\usecolortheme{wolverine}

    %\setbeamertemplate{footline} % To remove the footer line in all slides uncomment this line
    %\setbeamertemplate{footline}[page number] % To replace the footer line in all slides with a simple slide count uncomment this line

    %\setbeamertemplate{navigation symbols}{} % To remove the navigation symbols from the bottom of all slides uncomment this line
}

\usepackage{graphicx} % Allows including images
\usepackage{booktabs} % Allows the use of \toprule, \midrule and \bottomrule in tables

%----------------------------------------------------------------------------------------
%	TITLE PAGE
%----------------------------------------------------------------------------------------

\title[Procedural Terrain Generation]{Procedural Terrain Generation with Marching Cubes} % The short title appears at the bottom of every slide, the full title is only on the title page

\author{Michael Schenck} % Your name
    %University of California \\ % Your institution for the title page
    \medskip
    %\textit{john@smith.com} % Your email address
}
\date{\today} % Date, can be changed to a custom date

\begin{document}

\begin{frame}
    \titlepage % Print the title page as the first slide
\end{frame}

\begin{frame}
    \frametitle{Overview} % Table of contents slide, comment this block out to remove it
    \tableofcontents % Throughout your presentation, if you choose to use \section{} and \subsection{} commands, these will automatically be printed on this slide as an overview of your presentation
\end{frame}

%----------------------------------------------------------------------------------------
%	PRESENTATION SLIDES
%----------------------------------------------------------------------------------------

%------------------------------------------------
\section{Motivation} % Sections can be created in order to organize your presentation into discrete blocks, all sections and subsections are automatically printed in the table of contents as an overview of the talk
%------------------------------------------------

\subsection{} % A subsection can be created just before a set of slides with a common theme to further break down your presentation into chunks

\begin{frame}
    \frametitle{Motivation}

    \begin{columns}[c] % The "c" option specifies centered vertical alignment while the "t" option is used for top vertical alignment

        \column{.4\textwidth} % Left column and width
        \textbf{Traditional Terrain}
        \begin{itemize}
            \item 2-D Height Fields
            \item Generated by CPU
            \item Rendered by GPU
        \end{itemize}

        \column{.7\textwidth} % Right column and width
            \begin{figure}
                \fbox{\includegraphics[scale=0.215]{Pictures/traditional_terrain.jpg}}
                \tiny{ \url{Source: http://www.outerra.com/procedural/k080-3.jpg}}
            \end{figure}


        \end{columns}


    \end{frame}

    %------------------------------------------------

    \begin{frame}
        \frametitle{Motivation}

        \begin{columns}[c] % The "c" option specifies centered vertical alignment while the "t" option is used for top vertical alignment

            \column{.4\textwidth} % Left column and width
            \textbf{Complex Terrain}
            \begin{itemize}
                \item Defined by 3D Density Fields (functions)  
                \item Generated and rendered by GPU (ideally) 
            \end{itemize}

        \column{.6\textwidth} % Right column and width
            \begin{figure}
                \fbox{\includegraphics[scale=0.16]{Pictures/complex_terrain.jpg}}
                \tiny{\url{Source: https://arches.liris.cnrs.fr/publications/images/2009_arches/Cote_Finale_2.jpg}}

            \end{figure}

            \end{columns}


        \end{frame}

        %------------------------------------------------
        \section{Marching Cubes}

        \begin{frame}
            \frametitle{Marching Cubes}
            \begin{itemize}
                \item Marching Cubes is an algorithm to visualize an isosurface surface defined by a density field.
                \item Developed to visual 3D medical data from MRT and CAT scans.
                \item Divides a volume into voxels, and computes geometry for the given voxel if the isosurface intersects it.
                \item A highly parallel task that can be efficiently computed on the GPU.
                \item Also works on the CPU (as here).
            \end{itemize}

        \end{frame}

        %------------------------------------------------

        \begin{frame}
            \frametitle{Marching Cubes}
            \begin{columns}[c] % The "c" option specifies centered vertical alignment while the "t" option is used for top vertical alignment

                \column{.45\textwidth} % Left column and width
                \textbf{2D Equivalent: Marching Squares}
                \begin{itemize}
                    \item Assuming an isosurface value of 0, the surface will pass through edges that share corners where the isosurface evaluates to opposite signs.
                    \item Determine location along edge by interpolating between the two corner values.
                \end{itemize}

                \column{.5\textwidth} % Right column and width
                \begin{figure}
                    \includegraphics[scale=0.15]{Pictures/marchingsquares1.jpg}
                    \includegraphics[scale=0.15]{Pictures/marchingsquares2.jpg}
                    \tiny{\url{Source: http://www.cs.carleton.edu/cs_comps/0405/shape/marching_cubes.html}}

                \end{figure}

            \end{columns}
        \end{frame}

        %------------------------------------------------


        \begin{frame}
            \frametitle{Marching Cubes}
            \begin{columns}[c] % The "c" option specifies centered vertical alignment while the "t" option is used for top vertical alignment

                \column{.45\textwidth} % Left column and width
                \textbf{Marching Cubes Procedure:}
                \begin{enumerate}
                    \item Divide a volume into cubes -- or voxels.
                    \item Label each corner of the cubes. 
                    \item Each corner represents a bit
                \end{enumerate}

                \column{.5\textwidth} % Right column and width
                \begin{figure}
                    \includegraphics[scale=0.5]{Pictures/mcubes0.png}
                    \tiny{Source: Geiss, Nvidia GPU Gems}

                \end{figure}

            \end{columns}
        \end{frame}

        %------------------------------------------------


        \begin{frame}
            \frametitle{Marching Cubes}
            \begin{columns}[c] % The "c" option specifies centered vertical alignment while the "t" option is used for top vertical alignment

                \column{.45\textwidth} % Left column and width
                \begin{enumerate}
                    \setcounter{enumi}{3}
                \item Evaluate the density function at the $(x,y,z)$ of each corner.
                \item Assign a value of $0$ to coners where the density function is $< 0$ and a value of $1$ to those where the density function is $>0$.
                \item Concatenate the bits, and evaluate the resulting 8-bit number to determine the case number. 
                
                \end{enumerate}

                \column{.5\textwidth} % Right column and width
                \begin{figure}
                    \includegraphics[scale=0.5]{Pictures/mcubes1.png}
                    \tiny{Source: Geiss, Nvidia GPU Gems}

                \end{figure}

            \end{columns}
        \end{frame}
        %------------------------------------------------
        \begin{frame}
            \frametitle{Marching Cubes}
            \begin{columns}[c] % The "c" option specifies centered vertical alignment while the "t" option is used for top vertical alignment

                \column{.45\textwidth} % Left column and width
                \begin{enumerate}
                    \setcounter{enumi}{6}
                \item The case number acts as an index to two look-up tables (LUTs)
                \item \textbf{Table 1.} returns which edges are intersected by the isosurface.   
                \item The field values at the corner of the returned edges can then be interpolated to determine an approximation of the isosurface intersection along the edge.
                \end{enumerate}

                \column{.5\textwidth} % Right column and width
                \begin{figure}
                    \includegraphics[scale=0.5]{Pictures/mcubes2.png}
                    \tiny{Source: Geiss, Nvidia GPU Gems}

                \end{figure}

            \end{columns}
        \end{frame}
        %------------------------------------------------

        \begin{frame}
            \frametitle{Marching Cubes}
            \begin{columns}[c] % The "c" option specifies centered vertical alignment while the "t" option is used for top vertical alignment

                \column{.45\textwidth} % Left column and width
                \begin{enumerate}
                    \setcounter{enumi}{9}
                \item Given the intersection locations, triangles can be used to approximate the isosurface.
                \item \textbf{Table 2.} is a sequence table that returns groups of 3 edges to connect to form triangles.
               \item The triangle vertices are placed at the approximated intersection locations along each edge.
               \end{enumerate}

                \column{.5\textwidth} % Right column and width
                \begin{figure}
                    \includegraphics[scale=0.5]{Pictures/mcubes3.png}\\
                    \tiny{Source: Geiss, Nvidia GPU Gems}

                \end{figure}

            \end{columns}
        \end{frame}
        %------------------------------------------------
        \begin{frame}
            \frametitle{Marching Cubes}
            \begin{itemize}
                \item Note that for \textbf{case 0} and \textbf{case 255} all corners were negative or positive and no geometry needs to be generated.
                 \item This process is parallel, and through the use of LUT, fast.
                 \item By adjusting the density of cubes used to render a density field, the compution cost and resolution can be controlled.
                 
            \end{itemize}

        \end{frame}

        %------------------------------------------------
        \begin{frame}
            \frametitle{Marching Cubes}
            \begin{columns}[c] % The "c" option specifies centered vertical alignment while the "t" option is used for top vertical alignment

                \column{.45\textwidth} % Left column and width

                \begin{figure}
                    \includegraphics[scale=0.4]{Pictures/sphere_lowres.png}\\
                    \small{Sphere generated with \textbf{3x3x3} grid}

                \end{figure}
                \column{.5\textwidth} % Right column and width
                \begin{figure}
                    \includegraphics[scale=0.4]{Pictures/sphere_highres.png}\\
                    \small{Sphere generated with \textbf{10x10x10} grid}

                \end{figure}
            \end{columns}
        \end{frame}

        %------------------------------------------------
        \section{Second Section}
        %------------------------------------------------

        \begin{frame}
            \frametitle{Table}
            \begin{table}
                \begin{tabular}{l l l}
                    \toprule
                    \textbf{Treatments} & \textbf{Response 1} & \textbf{Response 2}\\
                    \midrule
                    Treatment 1 & 0.0003262 & 0.562 \\
                    Treatment 2 & 0.0015681 & 0.910 \\
                    Treatment 3 & 0.0009271 & 0.296 \\
                    \bottomrule
                \end{tabular}
                \caption{Table caption}
            \end{table}
        \end{frame}

        %------------------------------------------------

        \begin{frame}
            \frametitle{Theorem}
            \begin{theorem}[Mass--energy equivalence]
                $E = mc^2$
            \end{theorem}
        \end{frame}

        %------------------------------------------------

        \begin{frame}[fragile] % Need to use the fragile option when verbatim is used in the slide
            \frametitle{Verbatim}
            \begin{example}[Theorem Slide Code]
                \begin{verbatim}
                \begin{frame}
                    \frametitle{Theorem}
                    \begin{theorem}[Mass--energy equivalence]
                        $E = mc^2$
                    \end{theorem}
                \end{frame}\end{verbatim}
                \end{example}
            \end{frame}

            %------------------------------------------------

            \begin{frame}
                \frametitle{Figure}
                Uncomment the code on this slide to include your own image from the same directory as the template .TeX file.
                %\begin{figure}
                %\includegraphics[width=0.8\linewidth]{test}
                %\end{figure}
            \end{frame}

            %------------------------------------------------

            \begin{frame}[fragile] % Need to use the fragile option when verbatim is used in the slide
                \frametitle{Citation}
                An example of the \verb|\cite| command to cite within the presentation:\\~

                This statement requires citation \cite{p1}.
            \end{frame}

            %------------------------------------------------

            \begin{frame}
                \frametitle{References}
                \footnotesize{
                    \begin{thebibliography}{99} % Beamer does not support BibTeX so references must be inserted manually as below
                        \bibitem[Smith, 2012]{p1} John Smith (2012)
                            \newblock Title of the publication
                            \newblock \emph{Journal Name} 12(3), 45 -- 678.
                    \end{thebibliography}
                }
            \end{frame}

            %------------------------------------------------

            \begin{frame}
                \Huge{\centerline{The End}}
            \end{frame}

            %----------------------------------------------------------------------------------------

        \end{document} 
